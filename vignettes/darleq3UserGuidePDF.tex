\documentclass[]{article}
\usepackage{lmodern}
\usepackage{amssymb,amsmath}
\usepackage{ifxetex,ifluatex}
\usepackage{fixltx2e} % provides \textsubscript
\ifnum 0\ifxetex 1\fi\ifluatex 1\fi=0 % if pdftex
  \usepackage[T1]{fontenc}
  \usepackage[utf8]{inputenc}
  \usepackage{eurosym}
\else % if luatex or xelatex
  \ifxetex
    \usepackage{mathspec}
  \else
    \usepackage{fontspec}
  \fi
  \defaultfontfeatures{Ligatures=TeX,Scale=MatchLowercase}
  \newcommand{\euro}{€}
\fi
% use upquote if available, for straight quotes in verbatim environments
\IfFileExists{upquote.sty}{\usepackage{upquote}}{}
% use microtype if available
\IfFileExists{microtype.sty}{%
\usepackage{microtype}
\UseMicrotypeSet[protrusion]{basicmath} % disable protrusion for tt fonts
}{}
\usepackage[margin=1in]{geometry}
\usepackage{hyperref}
\hypersetup{unicode=true,
            pdftitle={darleq3: User Guide (Version 0.8.4)},
            pdfauthor={Steve Juggins and Martyn Kelly},
            pdfborder={0 0 0},
            breaklinks=true}
\urlstyle{same}  % don't use monospace font for urls
\usepackage{color}
\usepackage{fancyvrb}
\newcommand{\VerbBar}{|}
\newcommand{\VERB}{\Verb[commandchars=\\\{\}]}
\DefineVerbatimEnvironment{Highlighting}{Verbatim}{commandchars=\\\{\}}
% Add ',fontsize=\small' for more characters per line
\usepackage{framed}
\definecolor{shadecolor}{RGB}{248,248,248}
\newenvironment{Shaded}{\begin{snugshade}}{\end{snugshade}}
\newcommand{\KeywordTok}[1]{\textcolor[rgb]{0.13,0.29,0.53}{\textbf{#1}}}
\newcommand{\DataTypeTok}[1]{\textcolor[rgb]{0.13,0.29,0.53}{#1}}
\newcommand{\DecValTok}[1]{\textcolor[rgb]{0.00,0.00,0.81}{#1}}
\newcommand{\BaseNTok}[1]{\textcolor[rgb]{0.00,0.00,0.81}{#1}}
\newcommand{\FloatTok}[1]{\textcolor[rgb]{0.00,0.00,0.81}{#1}}
\newcommand{\ConstantTok}[1]{\textcolor[rgb]{0.00,0.00,0.00}{#1}}
\newcommand{\CharTok}[1]{\textcolor[rgb]{0.31,0.60,0.02}{#1}}
\newcommand{\SpecialCharTok}[1]{\textcolor[rgb]{0.00,0.00,0.00}{#1}}
\newcommand{\StringTok}[1]{\textcolor[rgb]{0.31,0.60,0.02}{#1}}
\newcommand{\VerbatimStringTok}[1]{\textcolor[rgb]{0.31,0.60,0.02}{#1}}
\newcommand{\SpecialStringTok}[1]{\textcolor[rgb]{0.31,0.60,0.02}{#1}}
\newcommand{\ImportTok}[1]{#1}
\newcommand{\CommentTok}[1]{\textcolor[rgb]{0.56,0.35,0.01}{\textit{#1}}}
\newcommand{\DocumentationTok}[1]{\textcolor[rgb]{0.56,0.35,0.01}{\textbf{\textit{#1}}}}
\newcommand{\AnnotationTok}[1]{\textcolor[rgb]{0.56,0.35,0.01}{\textbf{\textit{#1}}}}
\newcommand{\CommentVarTok}[1]{\textcolor[rgb]{0.56,0.35,0.01}{\textbf{\textit{#1}}}}
\newcommand{\OtherTok}[1]{\textcolor[rgb]{0.56,0.35,0.01}{#1}}
\newcommand{\FunctionTok}[1]{\textcolor[rgb]{0.00,0.00,0.00}{#1}}
\newcommand{\VariableTok}[1]{\textcolor[rgb]{0.00,0.00,0.00}{#1}}
\newcommand{\ControlFlowTok}[1]{\textcolor[rgb]{0.13,0.29,0.53}{\textbf{#1}}}
\newcommand{\OperatorTok}[1]{\textcolor[rgb]{0.81,0.36,0.00}{\textbf{#1}}}
\newcommand{\BuiltInTok}[1]{#1}
\newcommand{\ExtensionTok}[1]{#1}
\newcommand{\PreprocessorTok}[1]{\textcolor[rgb]{0.56,0.35,0.01}{\textit{#1}}}
\newcommand{\AttributeTok}[1]{\textcolor[rgb]{0.77,0.63,0.00}{#1}}
\newcommand{\RegionMarkerTok}[1]{#1}
\newcommand{\InformationTok}[1]{\textcolor[rgb]{0.56,0.35,0.01}{\textbf{\textit{#1}}}}
\newcommand{\WarningTok}[1]{\textcolor[rgb]{0.56,0.35,0.01}{\textbf{\textit{#1}}}}
\newcommand{\AlertTok}[1]{\textcolor[rgb]{0.94,0.16,0.16}{#1}}
\newcommand{\ErrorTok}[1]{\textcolor[rgb]{0.64,0.00,0.00}{\textbf{#1}}}
\newcommand{\NormalTok}[1]{#1}
\usepackage{graphicx,grffile}
\makeatletter
\def\maxwidth{\ifdim\Gin@nat@width>\linewidth\linewidth\else\Gin@nat@width\fi}
\def\maxheight{\ifdim\Gin@nat@height>\textheight\textheight\else\Gin@nat@height\fi}
\makeatother
% Scale images if necessary, so that they will not overflow the page
% margins by default, and it is still possible to overwrite the defaults
% using explicit options in \includegraphics[width, height, ...]{}
\setkeys{Gin}{width=\maxwidth,height=\maxheight,keepaspectratio}
\IfFileExists{parskip.sty}{%
\usepackage{parskip}
}{% else
\setlength{\parindent}{0pt}
\setlength{\parskip}{6pt plus 2pt minus 1pt}
}
\setlength{\emergencystretch}{3em}  % prevent overfull lines
\providecommand{\tightlist}{%
  \setlength{\itemsep}{0pt}\setlength{\parskip}{0pt}}
\setcounter{secnumdepth}{0}
% Redefines (sub)paragraphs to behave more like sections
\ifx\paragraph\undefined\else
\let\oldparagraph\paragraph
\renewcommand{\paragraph}[1]{\oldparagraph{#1}\mbox{}}
\fi
\ifx\subparagraph\undefined\else
\let\oldsubparagraph\subparagraph
\renewcommand{\subparagraph}[1]{\oldsubparagraph{#1}\mbox{}}
\fi

%%% Use protect on footnotes to avoid problems with footnotes in titles
\let\rmarkdownfootnote\footnote%
\def\footnote{\protect\rmarkdownfootnote}

%%% Change title format to be more compact
\usepackage{titling}

% Create subtitle command for use in maketitle
\newcommand{\subtitle}[1]{
  \posttitle{
    \begin{center}\large#1\end{center}
    }
}

\setlength{\droptitle}{-2em}
  \title{darleq3: User Guide (Version 0.8.4)}
  \pretitle{\vspace{\droptitle}\centering\huge}
  \posttitle{\par}
  \author{Steve Juggins and Martyn Kelly}
  \preauthor{\centering\large\emph}
  \postauthor{\par}
  \predate{\centering\large\emph}
  \postdate{\par}
  \date{2018-06-15}


\begin{document}
\maketitle

\subsection{1. Introduction}\label{introduction}

\texttt{darleq3} is an R package for the assessment of river and lake
ecological status using diatom data obtained by light microscopy (LM) or
Next Generation Sequencing (NGS). The package contains functions to
import diatom and associated environmental data from Excel worksheets,
perform simple data validation checks, calculate various water quality
metrics, EQRs and Water Framework Directive (WFD) quality classes for
samples, and classification uncertainty for sites. The package can
calculate Trophic Diatom Index TDI5LM, TDI4 and TDI3 scores for light
microscopy river diatom samples, TDI5NGS for NGS river diatom samples,
Lake Trophic Diatom Index LTDI2 and LTDI1 scores for light microscopy
lake diatom samples, and Diatom Acidification Metric (DAM) scores for
lake and river light microscopy samples. Details of the TDI / LTDI
metrics, algorithm and derivation of the status class boundaries for
rivers are given in Kelly \emph{et al}. (2008) and for lakes in Bennion
\emph{et al}. (2014). Details of the DAM acidification metric is
described in Juggins \emph{et al}. (2016). Calculation of uncertainty of
classification is described in Kelly \emph{et al}. 2009.

\texttt{darleq3} can be run in two ways, either as an interactive shiny
app, or a a series of R functions issues from the R console or an R
script. The first method attempts to mimic the old DARLEQ2 software will
be the easiest for most users. The second methods will be more
convenient for processing multiple data sets, for automating darleq
calculations, or including them in a longer chain of analysis.

\subsection{2. Installation}\label{installation}

The easiest way to install \texttt{darleq3} is from a github repository.
To do this first install the package \texttt{devtools} with the
following command, omitting the prompt (``\textgreater{}''):

\begin{Shaded}
\begin{Highlighting}[]
\OperatorTok{>}\StringTok{ }\KeywordTok{install.packages}\NormalTok{(}\StringTok{"devtools"}\NormalTok{)}
\end{Highlighting}
\end{Shaded}

Then install \texttt{darleq3}. Note that this will also automatically
install some additional packages on which \texttt{darleq3} depends.

\begin{Shaded}
\begin{Highlighting}[]
\OperatorTok{>}\StringTok{ }\KeywordTok{library}\NormalTok{(devtools)}
\OperatorTok{>}\StringTok{ }\KeywordTok{install_github}\NormalTok{(}\StringTok{"nsj3/darleq3"}\NormalTok{, }\DataTypeTok{build_vignettes=}\OtherTok{TRUE}\NormalTok{)}
\end{Highlighting}
\end{Shaded}

\texttt{darleq3} also contains an example Excel data file. This can be
made available in a R session with the following:

\begin{Shaded}
\begin{Highlighting}[]
\OperatorTok{>}\StringTok{ }\KeywordTok{library}\NormalTok{(darleq3)}
\OperatorTok{>}\StringTok{ }\NormalTok{fn <-}\StringTok{ }\KeywordTok{system.file}\NormalTok{(}\StringTok{"extdata/DARLEQ2TestData.xlsx"}\NormalTok{, }\DataTypeTok{package=}\StringTok{"darleq3"}\NormalTok{)}
\end{Highlighting}
\end{Shaded}

The file can be opened in Excel using t he following:

\begin{Shaded}
\begin{Highlighting}[]
\OperatorTok{>}\StringTok{ }\CommentTok{# note running the following lines will open the file in Excel (if installed)}
\ErrorTok{>}\StringTok{ }\KeywordTok{shell.exec}\NormalTok{(fn)}
\end{Highlighting}
\end{Shaded}

\subsection{\texorpdfstring{3. Using the \texttt{darleq3} Shiny
app}{3. Using the darleq3 Shiny app}}\label{using-the-darleq3-shiny-app}

\texttt{darleq3} can run on a remote Shiny server or locally on a
desktop PC running RStudio. The app will function in exactly the same
way in both situations. To run \texttt{darleq3} on a remote shiny server
open a web browser and point it to either:

\url{https://nsj3.shinyapps.io/darleq3/}

\url{http://gpsgpuserver.ncl.ac.uk:3838/darleq3/}

Both these hosts have been set up for testing purposes and may change.
There may be problems running the app on the second server listed from
within the EA.

To run the app locally, simply start RStudio, load the \texttt{darleq3}
package and run the command \texttt{runDARLEQ()}:

\begin{Shaded}
\begin{Highlighting}[]
\OperatorTok{>}\StringTok{ }\KeywordTok{library}\NormalTok{(darleq3)}
\OperatorTok{>}\StringTok{ }\KeywordTok{runDARLEQ}\NormalTok{()}
\end{Highlighting}
\end{Shaded}

This should open a browser and display the DARLEQ3 shiny app.

\begin{figure}
\centering
\includegraphics{images/darleq3shiny.png}
\caption{darleq3 shiny app}
\end{figure}

To use the app follow these simple steps:

\begin{itemize}
\item
  1: Click the Browse\ldots{} button to select and upload a DARLEQ
  diatom file (see below).
\item
  2: Once uploaded, select a sheet and click import. A summary (number
  of samples \& taxa) will be displayed in the Data summary box when
  upload is complete.
\item
  3: Select the metric type. ``TDI3 \& 4 for LM'' will calculate TDI3
  and TDI4 for river samples according to the DARLEQ 2 taxon list,
  TDI5LM will calculate TDI5 for river LM diatom data, TDI for NGS will
  calculate TDI5NGS for river NGS diatom data, ``LTDI for LM'' will
  calculate LTDI1 and LTDI2 for lake LM data, and ``DAM for LM'' will
  calculate the diatom acidification metric for river LM data. A summary
  of results will appear in the Results summary box when the
  calculations are complete.
\item
  4: Click Download Results to save the results in an Excel file. The
  default name for this file will be the ``DARLEQ3\_Results\_''
  concatenated with the original data filename, worksheet name, and
  date.
\end{itemize}

To quit the app simple close the browser and or hit Escape in the
RStudio Console window.

\subsection{\texorpdfstring{4. Using the \texttt{darleq3} R
package}{4. Using the darleq3 R package}}\label{using-the-darleq3-r-package}

\texttt{darleq3} contains a number of functions for importing diatom
data, calculating various sample and site-based metrics, EQRs and WFD
quality classes, and saving the results in Excel format. The main
functions are:

\begin{itemize}
\tightlist
\item
  \texttt{darleq} import diatom data from an Excel file, calculate
  metrics, EQRs and WFD quality classes, and save results in Excel
  format
\item
  \texttt{read\_DARLEQ} import diatom data from an Excel file
\item
  \texttt{save\_DARLEQ} save metric and EQR results in an Excel file
\item
  \texttt{calc\_Metric\_EQR} calculate EQRS, WFD quality classes and
  summary diagnostic measures for multiple metrics
\item
  \texttt{calc\_Metric} calculate various diatom water quality metrics
\item
  \texttt{calc\_EQR} calculate sample and site EQRs and WFD quality
  classes
\item
  \texttt{runDARLEQ} run DARLEQ3 as an interactive shiny app in a web
  browser
\end{itemize}

Type ?function\_Name at the R prompt to get help and example usage for
these functions.

The \texttt{darleq3} functions have been designed to allow the user to
perform individual steps of the data analysis sequence individually, for
example importing diatom data, calculating a particular metric diatom
from LM or NGS diatom data, or calculating EQRs from a metric and site
information. These low-level functions are useful for embedding darleq3
in a longer data analysis chain using R. The package also includes
``wrapper'' functions, that ``wrap'' multiple low-level functions to
perform a complete analysis with a single function call.

\subsubsection{\texorpdfstring{4.1 \texttt{darleq3} wrapper
functions}{4.1 darleq3 wrapper functions}}\label{darleq3-wrapper-functions}

The most useful wrapper function is \texttt{darleq}. This function
imports data form an Excel file, calculates multiple metrics, EQRs and
WFD classes and saves the results to another Excel file in one step.

\begin{Shaded}
\begin{Highlighting}[]
\OperatorTok{>}\StringTok{ }\NormalTok{fn <-}\StringTok{ }\KeywordTok{system.file}\NormalTok{(}\StringTok{"extdata/DARLEQ2TestData.xlsx"}\NormalTok{, }\DataTypeTok{package=}\StringTok{"darleq3"}\NormalTok{)}
\OperatorTok{>}\StringTok{ }\KeywordTok{darleq}\NormalTok{(fn)}
\end{Highlighting}
\end{Shaded}

\texttt{darleq} will, by default, import data from the first sheet in
the Excel file, and calculate TDI3, TDI4 and TDI5LM. If the output
filename is not given the function will generate a name by concatenating
``DARLEQ3\_Results\_'' with the original filename, the sheet name and
the current date.

To specify the sheet name, a different metric, and a output file name:

\begin{Shaded}
\begin{Highlighting}[]
\OperatorTok{>}\StringTok{ }\NormalTok{fn <-}\StringTok{ }\KeywordTok{system.file}\NormalTok{(}\StringTok{"extdata/DARLEQ2TestData.xlsx"}\NormalTok{, }\DataTypeTok{package=}\StringTok{"darleq3"}\NormalTok{)}
\OperatorTok{>}\StringTok{ }\KeywordTok{darleq}\NormalTok{(fn, }\DataTypeTok{sheet=}\StringTok{"Lakes LTDI Test Data"}\NormalTok{, }\DataTypeTok{metrics=}\StringTok{"LTDI2"}\NormalTok{, }\DataTypeTok{outFile=}\StringTok{"Results.xlsx"}\NormalTok{)}
\end{Highlighting}
\end{Shaded}

To calculate and save results for multiple metrics:

\begin{Shaded}
\begin{Highlighting}[]
\OperatorTok{>}\StringTok{ }\NormalTok{fn <-}\StringTok{ }\KeywordTok{system.file}\NormalTok{(}\StringTok{"extdata/DARLEQ2TestData.xlsx"}\NormalTok{, }\DataTypeTok{package=}\StringTok{"darleq3"}\NormalTok{)}
\OperatorTok{>}\StringTok{ }\KeywordTok{darleq}\NormalTok{(fn, }\DataTypeTok{sheet=}\StringTok{"Lakes LTDI Test Data"}\NormalTok{, }\DataTypeTok{metrics=}\KeywordTok{c}\NormalTok{(}\StringTok{"LTDI1"}\NormalTok{, }\StringTok{"LTDI2"}\NormalTok{), }\DataTypeTok{outFile=}\StringTok{"Results.xlsx"}\NormalTok{)}
\end{Highlighting}
\end{Shaded}

\subsubsection{\texorpdfstring{4.2 \texttt{darleq3} low-level
functions}{4.2 darleq3 low-level functions}}\label{darleq3-low-level-functions}

\texttt{darleq3} low-level functions are useful for calculating partial
results or for embedding darleq3 in a longer data analysis sequence. The
key functions are \texttt{read\_DARLEQ} to import data from a
DARLEQ-formatted data file (see Section 6 below for guidelines on how to
format the data correctly). \texttt{read\_DARLEQ} returns a list with
two elements: \texttt{diatom\_data} - a data frame of the diatom count
or relative abundance data, and \texttt{header} - a data frame of
sample, site and environmental data from the header of the Excel file.

\begin{Shaded}
\begin{Highlighting}[]
\OperatorTok{>}\StringTok{ }\NormalTok{fn <-}\StringTok{ }\KeywordTok{system.file}\NormalTok{(}\StringTok{"extdata/DARLEQ2TestData.xlsx"}\NormalTok{, }\DataTypeTok{package=}\StringTok{"darleq3"}\NormalTok{)}
\OperatorTok{>}\StringTok{ }\NormalTok{d <-}\StringTok{ }\KeywordTok{read_DARLEQ}\NormalTok{(fn, }\StringTok{"Rivers TDI Test Data"}\NormalTok{)}
\OperatorTok{>}\StringTok{ }\KeywordTok{head}\NormalTok{(d}\OperatorTok{$}\NormalTok{diatom_data[, }\DecValTok{1}\OperatorTok{:}\DecValTok{8}\NormalTok{])}
\end{Highlighting}
\end{Shaded}

\begin{verbatim}
##          AC023A AC083A AC143A AC161A AC9999  AD009A   AM001A  AM004A
## SPR001 0.000000      0      0      0      0 0.31348 0.000000 0.00000
## AUT001 0.000000      0      0      0      0 0.00000 0.000000 0.00000
## SPR002 0.332226      0      0      0      0 0.00000 0.332226 0.00000
## AUT002 0.000000      0      0      0      0 0.00000 0.000000 0.00000
## SPR003 0.000000      0      0      0      0 0.00000 0.000000 0.00000
## AUT003 0.317460      0      0      0      0 0.00000 0.000000 0.31746
\end{verbatim}

\begin{Shaded}
\begin{Highlighting}[]
\OperatorTok{>}\StringTok{ }\KeywordTok{head}\NormalTok{(d}\OperatorTok{$}\NormalTok{header)}
\end{Highlighting}
\end{Shaded}

\begin{verbatim}
##        SampleID SiteID SAMPLE_DATE Alkalinity   Stream                             Reach
## SPR001   SPR001  36082  2010-04-14        242   KENNET            Hambridge Rd., Newbury
## AUT001   AUT001  36082  2004-09-21        242   KENNET            Hambridge Rd., Newbury
## SPR002   SPR002  34649  2004-04-02        408 LAMBOURN                        A4 Newbury
## AUT002   AUT002  34649  2004-09-21        408 LAMBOURN                        A4 Newbury
## SPR003   SPR003  36073  2004-04-02        213 LAMBOURN At Gauging Station, East Shefford
## AUT003   AUT003  36073  2004-09-21        213 LAMBOURN At Gauging Station, East Shefford
\end{verbatim}

\texttt{calc\_Metric\_EQR} calculates one or more diatom metrics and the
corresponding sample and site EQRS and WFD classes, and class
uncertainties. The function returns a list with an element for each
metric. Each element is itself a list containing sample EQRs, site EQRs
and uncertainties and a job summary.

\begin{Shaded}
\begin{Highlighting}[]
\OperatorTok{>}\StringTok{ }\NormalTok{fn <-}\StringTok{ }\KeywordTok{system.file}\NormalTok{(}\StringTok{"extdata/DARLEQ2TestData.xlsx"}\NormalTok{, }\DataTypeTok{package=}\StringTok{"darleq3"}\NormalTok{)}
\OperatorTok{>}\StringTok{ }\NormalTok{d <-}\StringTok{ }\KeywordTok{read_DARLEQ}\NormalTok{(fn, }\StringTok{"Rivers TDI Test Data"}\NormalTok{)}
\OperatorTok{>}\StringTok{ }\NormalTok{results <-}\StringTok{ }\KeywordTok{calc_Metric_EQR}\NormalTok{(d, }\DataTypeTok{metrics=}\KeywordTok{c}\NormalTok{(}\StringTok{"TDI4"}\NormalTok{, }\StringTok{"TDI5LM"}\NormalTok{))}
\OperatorTok{>}\StringTok{ }\KeywordTok{head}\NormalTok{(results}\OperatorTok{$}\NormalTok{TDI5LM}\OperatorTok{$}\NormalTok{EQR[, }\DecValTok{9}\OperatorTok{:}\DecValTok{15}\NormalTok{])}
\end{Highlighting}
\end{Shaded}

\begin{verbatim}
##        N_TDI5LM N2_TDI5LM Max_TDI5LM   TDI5LM  eTDI5LM EQR_TDI5LM Class_TDI5LM
## SPR001       41     12.39      16.93 55.60483 68.49447  1.0000000         High
## AUT001       23      3.09      51.57 70.01608 68.49447  0.7613627         Good
## SPR002       55     13.14      20.93 70.66666 69.28261  0.7639539         Good
## AUT002       39      7.69      26.91 66.08840 69.28261  0.8831896         High
## SPR003       39      9.37      27.30 49.94616 65.48689  1.0000000         High
## AUT003       32      5.87      37.14 39.65652 65.48689  1.0000000         High
\end{verbatim}

\begin{Shaded}
\begin{Highlighting}[]
\OperatorTok{>}\StringTok{ }\KeywordTok{head}\NormalTok{(results}\OperatorTok{$}\NormalTok{TDI5LM}\OperatorTok{$}\NormalTok{Uncertainty)}
\end{Highlighting}
\end{Shaded}

\begin{verbatim}
##    SiteID N  EQR WFDClass   CoCH  CoCG  CoCM  CoCP CoCB   ROM  CoCHG CoCMPB ROM_GM
## 43  36082 2 0.88     High  79.41 19.01  1.52  0.06 0.00 20.59  98.42   1.58   1.58
## 33  34649 2 0.82     High  58.61 38.23  3.09  0.07 0.00 41.39  96.84   3.16   3.16
## 42  36073 2 1.00     High 100.00  0.00  0.00  0.00 0.00  0.00 100.00   0.00   0.00
## 41  35965 2 1.00     High 100.00  0.00  0.00  0.00 0.00  0.00 100.00   0.00   0.00
## 36  35101 1 0.48 Moderate   0.06 14.04 62.37 23.35 0.19 85.96  14.10  85.90  14.10
## 35  35075 2 0.71     Good  14.23 71.48 14.05  0.23 0.00 28.52  85.72  14.28  14.28
\end{verbatim}

\texttt{save\_DARLEQ} saves the output from \texttt{calc\_Metric\_EQR}
in an Excel file:

\begin{Shaded}
\begin{Highlighting}[]
\OperatorTok{>}\StringTok{ }\NormalTok{fn <-}\StringTok{ }\KeywordTok{system.file}\NormalTok{(}\StringTok{"extdata/DARLEQ2TestData.xlsx"}\NormalTok{, }\DataTypeTok{package=}\StringTok{"darleq3"}\NormalTok{)}
\OperatorTok{>}\StringTok{ }\NormalTok{d <-}\StringTok{ }\KeywordTok{read_DARLEQ}\NormalTok{(fn, }\StringTok{"Rivers TDI Test Data"}\NormalTok{)}
\OperatorTok{>}\StringTok{ }\NormalTok{results <-}\StringTok{ }\KeywordTok{calc_Metric_EQR}\NormalTok{(d, }\DataTypeTok{metrics=}\KeywordTok{c}\NormalTok{(}\StringTok{"TDI4"}\NormalTok{, }\StringTok{"TDI5LM"}\NormalTok{))}
\OperatorTok{>}\StringTok{ }\KeywordTok{save_DARLEQ}\NormalTok{(results, }\DataTypeTok{outFile=}\StringTok{"Results.xlsx"}\NormalTok{)}
\end{Highlighting}
\end{Shaded}

\texttt{calc\_Metric} calculates a single metric from a data frame of
diatom count or relative abundance data.

\begin{Shaded}
\begin{Highlighting}[]
\OperatorTok{>}\StringTok{ }\NormalTok{fn <-}\StringTok{ }\KeywordTok{system.file}\NormalTok{(}\StringTok{"extdata/DARLEQ2TestData.xlsx"}\NormalTok{, }\DataTypeTok{package=}\StringTok{"darleq3"}\NormalTok{)}
\OperatorTok{>}\StringTok{ }\NormalTok{d <-}\StringTok{ }\KeywordTok{read_DARLEQ}\NormalTok{(fn, }\StringTok{"Rivers TDI Test Data"}\NormalTok{)}
\OperatorTok{>}\StringTok{ }\NormalTok{x <-}\StringTok{ }\KeywordTok{calc_Metric}\NormalTok{(d}\OperatorTok{$}\NormalTok{diatom_data, }\DataTypeTok{metric=}\StringTok{"TDI4"}\NormalTok{)}
\OperatorTok{>}\StringTok{ }\KeywordTok{head}\NormalTok{(x}\OperatorTok{$}\NormalTok{Metric)}
\end{Highlighting}
\end{Shaded}

\begin{verbatim}
##            TDI4
## SPR001 52.22581
## AUT001 83.44051
## SPR002 70.61036
## AUT002 67.53959
## SPR003 49.97693
## AUT003 38.45844
\end{verbatim}

\texttt{calc\_EQR} calculates sample and site EQRS and WFD classes, and
class uncertainties from a list of sample metrics.

\begin{Shaded}
\begin{Highlighting}[]
\OperatorTok{>}\StringTok{ }\NormalTok{fn <-}\StringTok{ }\KeywordTok{system.file}\NormalTok{(}\StringTok{"extdata/DARLEQ2TestData.xlsx"}\NormalTok{, }\DataTypeTok{package=}\StringTok{"darleq3"}\NormalTok{)}
\OperatorTok{>}\StringTok{ }\NormalTok{d <-}\StringTok{ }\KeywordTok{read_DARLEQ}\NormalTok{(fn, }\StringTok{"Rivers TDI Test Data"}\NormalTok{)}
\OperatorTok{>}\StringTok{ }\NormalTok{x <-}\StringTok{ }\KeywordTok{calc_Metric}\NormalTok{(d}\OperatorTok{$}\NormalTok{diatom_data, }\DataTypeTok{metric=}\StringTok{"TDI4"}\NormalTok{)}
\OperatorTok{>}\StringTok{ }\NormalTok{eqr <-}\StringTok{ }\KeywordTok{calc_EQR}\NormalTok{(x, d}\OperatorTok{$}\NormalTok{header)}
\OperatorTok{>}\StringTok{ }\KeywordTok{head}\NormalTok{(eqr}\OperatorTok{$}\NormalTok{EQR[, }\DecValTok{9}\OperatorTok{:}\DecValTok{15}\NormalTok{])}
\end{Highlighting}
\end{Shaded}

\begin{verbatim}
##        N_TDI4 N2_TDI4 Max_TDI4     TDI4    eTDI4  EQR_TDI4 Class_TDI4
## SPR001     41   12.39    16.93 52.22581 68.49447 1.0000000       High
## AUT001     23    3.09    51.57 83.44051 68.49447 0.4204847   Moderate
## SPR002     54   13.05    20.93 70.61036 69.28261 0.7654202       Good
## AUT002     39    7.69    26.91 67.53959 69.28261 0.8453951       High
## SPR003     39    9.37    27.30 49.97693 65.48689 1.0000000       High
## AUT003     32    5.87    37.14 38.45844 65.48689 1.0000000       High
\end{verbatim}

\begin{Shaded}
\begin{Highlighting}[]
\OperatorTok{>}\StringTok{ }\KeywordTok{head}\NormalTok{(eqr}\OperatorTok{$}\NormalTok{Uncertainty)}
\end{Highlighting}
\end{Shaded}

\begin{verbatim}
##    SiteID N  EQR WFDClass   CoCH  CoCG  CoCM  CoCP CoCB   ROM  CoCHG CoCMPB ROM_GM
## 43  36082 2 0.71     Good  14.23 71.48 14.05  0.23 0.00 28.52  85.72  14.28  14.28
## 33  34649 2 0.81     High  54.37 42.03  3.54  0.07 0.00 45.63  96.39   3.61   3.61
## 42  36073 2 1.00     High 100.00  0.00  0.00  0.00 0.00  0.00 100.00   0.00   0.00
## 41  35965 2 1.00     High 100.00  0.00  0.00  0.00 0.00  0.00 100.00   0.00   0.00
## 36  35101 1 0.47 Moderate   0.04 11.90 61.95 25.89 0.23 88.10  11.93  88.07  11.93
## 35  35075 2 0.71     Good  14.23 71.48 14.05  0.23 0.00 28.52  85.72  14.28  14.28
\end{verbatim}

\subsection{\texorpdfstring{5. Understanding \texttt{darleq3}
output}{5. Understanding darleq3 output}}\label{understanding-darleq3-output}

The DARLEQ shiny app and R functions produce output that is similar in
structure and content to that produced by the DARLEQ2 program.
Specifically, the shiny app and functions \texttt{darleq} and
\texttt{save\_DARLEQ} save data in an Excel file with the following
content. For each metric, the output file will contain 3 worksheets,
named Code\_Job\_Summary, and Code\_Uncertainty (where Code is the code
for each metric). These three sheets contain the following information:

\subsubsection{5.1 Job summary}\label{job-summary}

This sheet contains the input file name, worksheet name and a summary of
the number of samples and taxa in the file. It also contains a list of
taxa included in the file but excludes from the metric calculations
either because they are planktic or because they are not included in the
DARLEQ list of indicator values for that metric. The list also contains
the number of occurrences (N), Hill's N2 effective number of occurrences
(Hill 1973) and maximum abundance of these taxa. The list of useful in
checking the data for coding errors to identify abundant taxa excluded
from the metric calculations. For TDI3/4 and LTDI1/2 the output also
contains a list of taxa with indicator values included in DARLEQ3 but
not in DARLEQ2 software. This is useful in understanding the reasons for
differences in metric scores between DARLEQ versions 2 and 3 for the
same metric.

\subsubsection{5.2 Sample\_Summary}\label{sample_summary}

Sample Summary â\euro{}`` this sheet contains metric, EQR and quality
class results for each sample. First, the sample information listed in
the original input file is repeated, and then results of the analysis
are listed as follows (where CODE is the metric Code):

\begin{itemize}
\tightlist
\item
  Total\_count: Sum of the counts or percentages of all taxa in a
  sample.
\end{itemize}

Percent\_in\_CODE: Percentages of the total count of taxa that are
matched to taxa in the master taxon list and included in the metric
calculations. If all taxa are matched this will be the same as the
Total\_count but will be less if, for example, planktic taxa are
present. Comparison of these two fields will indicate if there are
important taxa present in the sample but not included in the status
calculations.

\begin{itemize}
\item
  N\_CODE, N2\_CODE, Max\_CODE: Number of taxa (N), effective number of
  taxa (N2) and maximum abundance (max) of taxa included in the metric
  calculations.
\item
  CODE: value of the metric for each sample.
\item
  eCODE: Expected value of the metric for each sample according to
  typology (lakes) or site-specific prediction (rivers).
\item
  EQR\_CODE: EQR for each sample based on predicted and observed
  metrics.
\item
  Class\_CODE: Status class based on EQR.
\end{itemize}

After the metric and classification fields a series of summary fields
are listed containing the percentage of various ecological groups of
diatoms:

\begin{itemize}
\item
  Motile: Percentage of the motile diatoms in the sample.
\item
  OrganicTolerant: Percentage of organic pollution tolerant diatoms in
  the sample.
\item
  Planktic: Percentage of planktic diatoms in the sample. These are
  excluded from the status calculations.
\item
  Saline: Percentage of diatoms tolerant of slightly saline waters.
\item
  Comments: List of any warning messages generated during calculations
  for individual samples relating to missing or out-of-range
  environmental values.
\end{itemize}

\subsubsection{5.3 Uncertainty}\label{uncertainty}

Multiple samples from each site are combined and an uncertainty analysis
is performed using the mean EQR and number of samples according to Kelly
\emph{et al}. (2009):

\begin{itemize}
\item
  SiteID: Unique site code taken from row 2 of the input data.
\item
  N: Number of samples for site used in calculation of mean EQR and CoC.
\item
  EQR: Mean EQR for each site.
\item
  lake\_TYPE: lake type (only for lake data)
\item
  WFDClass: Status class based on mean EQR.
\item
  CoCH - CoCB: Confidence that the site belongs to status class high,
  good, etc.
\item
  RoM: Risk of misclassification for predicted class.
\item
  CoCHG: Confidence that the site is better than moderate class.
\item
  CoCMPB: Confidence that the site is moderate or worse class.
\item
  RoM\_GM: Rick of misclassification above / below the good / moderate
  boundary.
\end{itemize}

\subsection{6. Input data format}\label{input-data-format}

\texttt{read\_DARLEQ} and the shiny app import diatom data from an Excel
file in either .xls or .xlsx format. An example Excel file is included
in this package (see Section 2 on how to view it). The required data and
layout are rather and are slightly different for river and lake samples.
Figure 2 below shows the required format for performing TDI calculations
for river samples.

The first four header rows are mandatory and must contain the following
information:

\begin{itemize}
\item
  Row 1: SampleID: a short numerical or alphanumeric code to uniquely
  identify the sample. This field cannot be empty (an empty cell
  indicates the end of data).
\item
  Row 2: SiteID â\euro{}`` a short numerical or alphanumeric code to
  uniquely identify the site. This code will be used to aggregate
  multiple samples when calculating confidence of class for a site.
\item
  Row 3: SampleDate: sample date in Day/Month/Year format. Missing dates
  are set to Spring for the purposes of classification using TDI3 and
  samples flagged with a warning.
\item
  Row 4: Alkalinity: Mean annual alkalinity (or best available estimate)
  in mg l-1 (CaCO3). Missing values are set to 100 mg l-1 for the
  purposes of classification and samples flagged with a warning.
  Alkalinity values outside the range of the site prediction algorithm
  are set to the appropriate limit (6 or 150 mg l-1 for TDI3 and 5 or
  250 mg l\^{}-1 for TDI4 and TDI5LM / TDI5NGS).
\item
  Rows 5+: Further option sample descriptors such as river name, reach
  name etc. These data are not used by the program but will be
  reproduced in the output.
\end{itemize}

Note that the second column of the header information must be left
blank.

\begin{figure}
\centering
\includegraphics{images/DARLEQRiverData.png}
\caption{Example format for river diatom samples}
\end{figure}

Identifiers for each row of the sample header information should be
listed in column 1. Diatom data then follow the header information and
may be in count or percentage format. The first column must contain the
taxon code in either NBS or DiatCode
(\url{http://www.ecrc.ucl.ac.uk/?q=databases/diatcode}) format. The
codes in this column are used to link the data to the DARLEQ3 taxon list
and ecological information and cannot be empty (an empty cell indicates
the end of the data). The second column must include either the taxon
name or code (ie. a repeat of column 1).

The remaining columns to the right of the taxon name contain diatom
counts or percentages. Empty (blank) cells in the matrix will be read as
zero. Character data in the diatom matrix will generate an error. A full
list of diatom codes (either NBS or DiatCodes) are available in the data
frame \texttt{darleq3\_taxa}.

If the Diatom Acidification Metric (DAM) is to be calculated, the header
must contain estimates of mean annual Calcium and DOC concentrations,
rows named Calcium and DOC, and in ueq l-1 and mg l-1 respectively.
Figure 3 shows an example formatted for calculation of TDI and DAM. Note
that if only DAM scores are required the Alkalinity field may be left
blank. Sample Date is not used for calculating DAM and may be left
blank.

\begin{figure}
\centering
\includegraphics{images/DARLEQDAMData.png}
\caption{Example format for river diatom TDI and DAM samples}
\end{figure}

The required input format for lake samples is shown in Figure 4. This is
exactly the same as for river data except that the fourth row must be
named LAKE\_TYPE and contain a code indicating lake type according to
the GB lake typology alkalinity classes. Marl lakes are included in the
high alkalinity (HA) group. Peat and brackish lakes are not covered by
the tool. Sample date for lake samples is not used in the class
calculations and can contain missing values.

\begin{figure}
\centering
\includegraphics{images/DARLEQLakeData.png}
\caption{Example format for lake diatom LTDI samples}
\end{figure}

\subsection{7. Acknowledgements}\label{acknowledgements}

\subsection{8. References}\label{references}

Bennion, H., Kelly, M.G., Juggins, S., Yallop, M.L., Burgess, A.,
Jamieson, J., Krokowski, J., 2014. Assessment of ecological status in UK
lakes using benthic diatoms. \emph{Freshwater Science} \textbf{33},
639-654.

Juggins, S., Kelly, M., Allott, T., Kelly-Quinn, M., Monteith, D., 2016.
A Water Framework Directive-compatible metric for assessing
acidification in UK and Irish rivers using diatoms. \emph{Science of The
Total Environment} \textbf{568}, 671-678.

Kelly, M., Bennion, H., Burgess, A., Ellis, J., Juggins, S., Guthrie,
R., Jamieson, J., Adriaenssens, V., Yallop, M., 2009. Uncertainty in
ecological status assessments of lakes and rivers using diatoms.
\emph{Hydrobiologia} \textbf{633}, 5-15.

Kelly, M., Juggins, S., Guthrie, R., Pritchard, S., Jamieson, J.,
Rippey, B., Hirst, H., Yallop, M., 2008. Assessment of ecological status
in UK rivers using diatoms. \emph{Freshwater Biology} \textbf{53},
403-422.


\end{document}
